\documentclass[10pt,a4paper]{report}
\usepackage[utf8]{inputenc}
\usepackage{amsmath}
\usepackage{amsfonts}
\usepackage{amsthm}
\usepackage{amssymb}
\usepackage{polski}
\usepackage[left=2cm,right=2cm,top=2cm,bottom=2cm]{geometry}

\newtheorem{definition}[section]{Definicja}
\newtheorem{theorem} [section] {Twierdzenie]
\newtheorem{algorithm}[section]{Algorytm}

\newcommand{\transpose}[1]{{#1}^T}


\author{Anna Szczepaniak}
\title{Algorytm dekompozycji QR}
\begin{document}

\maketitle


\tableofcontents

\chapter{Wstęp}


\chapter{Preliminaria}

\subsection{Definicje dotyczące macierzy} 

\begin{definition}[Definicja macierzy]
Macierzą $m \times n$ (tzn. o m wierszach i n kolumnach) o wyrazach w zbiorze X nazywamy tablicę:
$$
\mathbf{X} =
\begin{bmatrix}
x_{11} & x_{12} & \ldots & x_{1n}&   \\
x_{21} & x_{22} & \ldots & x_{2n}& \\
\vdots & \vdots & \ddots & \vdots \\
x_{m1} & x_{m2} & \ldots & x_{mn}&
\end{bmatrix}
$$
gdzie $x_{ij} \in X$  dla $1\le i\le m$, $1\le j\le n$.
\end{definition}

Korzystając z definicji \ref{definicja-macierzy}

begin{definition} \label{definicja-macierzy}
Macierzą kwadratową o wymiarze n x n nazywamy macierz o liczbie wierszy równiej liczbie kolumn. Liczbę n nazywamy stopniem macierzy.
\end{definition}

begin{definition}
Macierzą jednostkową nazywamy macierz kwadratową, która na swojej głównej przekątnej posiada same wartości równe 1, natomiast reszta jest wypełniona zerami. Współczynniki tej macierzy są określone następującymi wzorami: 
$$
a_{ij} = \left\{ \begin{array}{ll}
1 & \textrm{dla $j=i$}\\
0 & \textrm{dla $j\ne i$}\\
\end{array} \right.
$$
\end{definition}

begin{definition}
Macierzą symetryczną nazywamy macierz kwadratową, której wyrazy położone symetrycznie względem głównej przekątnej są równe. Formalnie jest to macierz kwadratowa 
$$A=[a_{ij}]$$ 
stopnia n, która dla 
$$i,j=1,...,n$$ 
spełnia warunek 
$$
a_{ij}=a_{ji}.
$$
\end{definition}

begin{definition}
Macierz trójkątna to macierz kwadratowa, której wszystkie współczynniki pod główną przekątną lub wszystkie współczynniki nad tą przekątną są równe zero.
\end{definition}

begin{definition}
Macierzą transponowaną (przestawioną) macierzy A nazywamy macierz $A^{T},$ która powstaje z danej poprzez zamianę jej wierszy na kolumny i kolumn na wiersze. Operację tworzenia macierzy transponowanej nazywamy transpozycją (przestawianiem).
Dla macierzy $A=(a_{ij}):$ 
$$ 
\transpose{A} = \transpose{(a_{ij})}=(a_{ji})
$$.
\end{definition}

begin{definition}
Iloczynem macierzy $A = [a_{ij} ]$ i $B = [b_{jk}]$, gdzie $A\in M_{m}^{n}(F), $ $B\in M_{p}^{m}(F)$ nazywamy macierz $C = [c_{ik}], $ $C \in M_{p}^{n}(F), $ daną wzorem 
$$ c_{ik} = \sum_{j=1}^{n} a_{ij}\cdot b_{jk}. $$
Oznaczamy C = A · B.
\end{definition}

begin{definition}
Niech A będzie macierzą kwadratową ustalonego stopnia. Macierz A jest odwracalna, jeśli istnieje taka macierz B, że zachodzi
$$A\cdot B=B\cdot A=I, $$ 
gdzie I jest macierzą jednostkową. Macierz B nazywa się wówczas macierzą odwrotną do macierzy A i oznacza się przez  $A^{-1}.$
\end{definition}

begin{definition}
Macierz A nazywamy macierzą nieosobliwą, jeśli istnieje macierz B, która jest do niej odwrotna.
\end{definition}  

begin{definition}
Macierzą ortogonalną nazywamy macierz kwadratową $A\in M_{n}(R)$ o elementach będących liczbami rzeczywistymi spełniająca równość:
$$A^{T}\cdot A=A\cdot A^{T}=I_{n},$$ 
gdzie $I_{n}$ oznacza macierz jednostkową wymiaru n, $A^{T}$ oznacza macierz transponowaną względem A.
\end{definition}

begin{definition}
Dwie macierze kwadratowe A i B nazywamy macierzami podobnymi, jeśli istnieje taka macierz nieosobliwa P, że zachodzi związek: 
$$ B=P^{-1}\cdot A\cdot P. $$
\end{definition}


\subsection{Definicje dotyczące wektorów} 

begin{definition}
Wektorem nazywamy uporządkowaną parę punktów, z których jeden jest początkiem a drugi końcem tego wektora. Każdy wektor posiada zwrot, kierunek i długość.
\end{definition}

begin{definition}
Układ wektorów  $ (v_{1}, v_{2}, ..., v_{k}) $ w przestrzeni liniowej $ V $ nad ciałem $ K $ nazywamy liniowo niezależnym, jeśli z $ a_{1}\cdot v_{1}+ a_{2}\cdot v_{2} + ... + a_{k}\cdot v_{k} = 0 $ wynika, że $a_{1}=a_{2}=...=a_{k}=0 $.
\end{definition}

\subsection{Definicje dotyczące wyznacznika macierzy, jej wartości własnej oraz wektorów własnych.}

begin{definition}
Niech będzie dana macierz kwadratowa A stopnia n. Wyznacznikiem nazywamy takie odwzorowanie, które danej macierzy A wymiaru $n \times n$ przyporządkowuje dokładnie jedną liczbę rzeczywistą detA. Jeśli macierz jest stopnia $n = 1, $ to jej wyznacznik $detA = a_{11}$. 
Jeśli stopień macierzy jest większy niż 1, to jej wyznacznik obliczamy według następującego wzoru: 
$$
detA = \sum_{i=1}^{n} (-1)^{i+j}\cdot a_{ij}\cdot detM_{ij},
$$
gdzie $detM_{ij}$ oznacza wyznacznik macierzy powstałej z macierzy A przez skreślenie i-tego wiersza i j-tej kolumny.
\end{definition}
\newpage
begin{definition}
Niech B będzie macierzą wymiaru m x m i niech I będzie macierzą jednostkową wymiaru m x m. Wówczas skalary (liczby o wymiarze 1 x 1 ) $\lambda_{1}, \lambda_{2}, ..., \lambda_{m} $ nazywamy wartościami własnymi macierzy B, jeśli spełniają 
$$|B-\lambda\cdot I|=0. $$ 
\end{definition}

begin{definition}
Niech B będzie macierzą wymiaru m x m i niech $\lambda$ będzie wartością własną macierzy B. Wówczas niezerowy wektor e wymiaru m x 1 nazywamy wektorem własnym macierzy B, jeżeli 
$$B\cdot e = \lambda\cdot e. $$
\end{definition}











\chapter{Algorytm QR}

begin{theorem}[Twierdzienie]\label{(O rozkładzie QR)}
Niech A będzie macierzą o wymiarach m x n, gdzie $m\ge n$, której kolumny są liniowo niezależne. Istnieje wtedy jedyny rozkład A=QR na dwa czynniki: macierz Q o wymiarach m x n taka, że $Q^{T}\cdot Q=D,$ gdzie $D= diag (d_{1}, d_{2}, ..., d_{n}),$  $d_{k}>0$ dla $(k = 1, 2, ..., n)$ i macierz trójkątną górną R z elementami $r_{kk}= 1 (k = 1, 2, ..., n).$
\end{theorem}

\noindent \textbf{Dowód twierdzenia 3.1}\\
Podane wyżej twierdzenie jest przekształceniem
procesu ortogonalizacji Grama-Schmidta. Jeśli zastosujemy Grama-Schmidta do kolumn $a_{i}$ macierzy
 $A = [a_{1}, a_{2}, ..., a_{n}]$ od lewej do prawej, otrzymamy
sekwencje ortonormalnych wektorów od $q_{1}$ do $q_{n}$ obejmujących tę samą przestrzeń:
te ortogonalne wektory są kolumnami Q. Gram-Schmidt również wylicza
współczynniki wyrażające każdą kolumnę $a_{i}$
 jako liniową kombinację $q_{1}$ przez 
 $$q_{i}: a_{i}= \sum_{j=1}^i r_{ji}\cdot q_{i}, $$ 
 gdzie $r_{ji}$ to współczynniki macierzy R.
 



Algorytm QR został wynaleziony w 1961 roku przez Francisa i Kubłanowską. Jest jedną z efektywniejszych znanych metod rozwiązywania pełnego zadania własnego dla macierzy symetrycznych lub niesymetrycznych.
W podstawowym algorytmie QR tworzy się ciąg macierzy $A= A_{0}, A_{1}, A_{2}, ...$ taki, że 
$$ A_{s}=Q_{s}R_{s},$$
$$R_{s}Q_{s}=A_{s+1},$$
$$(s=0, 1, ...),$$ 
gdzie $Q_{s}$ jest macierzą ortogonalną, a $R_{s}$ - trójkątną górną. Łatwo widać, że z twierdzenia o rozkładzie QR wynika, że ciąg $A_{s}$ $(s=0, 1, ...)$ jest w zasadzie określony jednoznacznie. Ponieważ 
$$ A_{s+1}=R_{s}Q_{s}=Q_{s}^{T}A_{s}Q_{s},$$ 
więc każdy krok w algorytmie QR jest przekształceniem przez podobieństwo. 

\subsection*{Metoda Householdera}

Metoda Householdera pozwala znaleźć rozkład QR dowolnej macierzy prostokątnej m x n ($m\ge n$).

{Macierz Householdera}
Macierzą Householdera H zwaną również refleksją nazywamy symetryczną i ortogonalną macierz przekształcenia wektora, które odbija go względem pewnej płaszczyzny. 
\newpage
{Transformacja Householdera}
Niech  $v\in R^{m}, $ i $v\neq 0. $ Wówczas transformacją Householdera nazywamy macierz postaci:

$${\displaystyle H=I-Wvv^{T}, } {\displaystyle H=I-Wvv^{T}, }       {\displaystyle W={\frac {2}{v^{T}v}}} {\displaystyle W={\frac {2}{v^{T}v}}}$$ 
Macierz H jest macierzą symetryczną i ortogonalną oraz ma taką własność, że dowolny wektor x wymiaru m jest odbiciem lustrzanym wektora Hx względem hiperpłaszczyzny (wymiaru m-1) prostopadłej do wektora v[3]. Łatwo sprawdzić, że tak jest ponieważ: 

$${\displaystyle H^{2}=\left(I-{\frac {2vv^{T}}{v^{T}v}}\right)^{2}=I-{\frac {4vv^{T}}{v^{T}v}}+4\left({\frac {vv^{T}}{v^{T}v}}\right)^{2}=I} {\displaystyle H^{2}=\left(I-{\frac {2vv^{T}}{v^{T}v}}\right)^{2}}$$ \newline  $$= {I-{\frac {4vv^{T}}{v^{T}v}}+4\left({\frac {vv^{T}}{v^{T}v}}\right)^{2}=I}       {\displaystyle ((vv^{T})(v^{T}v)=(vv^{T})^{2})} {\displaystyle ((vv^{T})(v^{T}v)=(vv^{T})^{2})}$$\\
oraz\\

$${\displaystyle H^{T}=\left(I-{\frac {2vv^{T}}{v^{T}v}}\right)^{T}=I-\left({\frac {2vv^{T}}{v^{T}v}}\right)^{T}=I-{\frac {2vv^{T}}{v^{T}v}}=H} {\displaystyle H^{T}=\left(I-{\frac {2vv^{T}}{v^{T}v}}\right)^{T}}$$ \newline $$={I-\left({\frac {2vv^{T}}{v^{T}v}}\right)^{T}=I-{\frac {2vv^{T}}{v^{T}v}}=H}       {\displaystyle ((vv^{T})^{T}=(vv^{T}))} {\displaystyle ((vv^{T})^{T}=(vv^{T}))}$$\\
Z drugiej równości wynika symetria, z pierwszej ortogonalność, ponieważ $${\displaystyle H^{T}H=HH=I}. $$ Zatem:\\

$${\displaystyle |Hx|={\sqrt {(Hx)^{T}(Hx)}}={\sqrt {x^{T}(H^{T}H)x}}={\sqrt {x^{T}Ix}}=|x|} {\displaystyle |Hx|={\sqrt {(Hx)^{T}(Hx)}}={\sqrt {x^{T}(H^{T}H)x}}={\sqrt {x^{T}Ix}}=|x|}. $$\\
Mnożąc dowolny wektor ${\displaystyle x\in R^{m}}$ otrzymujemy:\\

$${\displaystyle Hx=x-{\frac {2vv^{T}x}{v^{T}v}}=x+(-2{\frac {vv^{T}x}{v^{T}v}})=x-2r} {\displaystyle Hx=x-{\frac {2vv^{T}x}{v^{T}v}}=x+(-2{\frac {vv^{T}x}{v^{T}v}})=x-2r}$$





\chapter{Eksperymenty numeryczne}

\chapter{Podsumowanie}

\end{document}