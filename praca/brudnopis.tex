\documentclass[10pt,a4paper]{report}
\usepackage[utf8]{inputenc}
\usepackage{amsmath}
\usepackage{amsfonts}
\usepackage{amsthm}
\usepackage{amssymb}
\usepackage{polski}
\usepackage[left=2cm,right=2cm,top=2cm,bottom=2cm]{geometry}




\begin{document}





Klasyczne algorytmy Gram-Schmidt (CGS) i zmodyfikowany GramSchmidt (MGS) dla dekompozycji A = QR:
for i = 1 do n 
$ q_{i} = a_{i}$ 
 for j = 1 to i - 1 
  $$\left\{ \begin{array}{ll}
   r_{ji} = q_{j}^{T}*a_{i} (CGS)  \\   
   r_{ji} = q_{j}^{T}*q_{i}  (MGS)   \\  
\end{array} \right.
$$
end for
$$
r_{ii} = \Arrowvert q_{i}\Arrowvert_{2}
$$
 
if 
$$r_{ii} = 0 $$
quit 
end if
 $$q_{i} = q_{i}/ r_{ii}$$
  end for
  
Niestety, CGS jest numerycznie niestabilny w arytmetyce zmiennoprzecinkowej, gdy kolumny A są prawie liniowo zależne. MGS jest bardziej stabilny, ale wciąż może powodować, że Q będzie daleko od ortogonalnego 
$$\Arrowvert Q^{T}Q - I\Arrowvert $$
. Algorytm 3.2 w sekcji 3.4.1 jest stabilnym alternatywnym algorytmem dla dekompozycji $A = QR$.  











Wyprowadzimy wzór na x, który minimalizuje 
$$\Arrowvert Ax - b\Arrowvert_{2}
$$
 używając dekompozycji 
 $$A = QR $$
 na trzy nieco inne sposoby. Po pierwsze, zawsze możemy wybrać m - n więcej ortonormalnych wektorów Q, tak że [Q, \~Q] jest kwadratową ortogonalną macierzą (na przykład możemy wybrać dowolne m - n więcej niezależnych wektorów X, które chcemy, a następnie zastosować Algorytm 3.1 do macierz nieosobliwej nxn [Q, X]).
 Wtedy
 $$ \Arrowvert Ax - b \Arrowvert_{2}^{2} =  \Arrowvert [Q,\~Q]^{T}(Ax - b)\Arrowvert_{2}^{2} 
 =
 \Arrowvert
(\mathbf{X} =
\begin{bmatrix}
Q^{T}\\
\~Q^{T}\\
\end{bmatrix})\cdot(Q RX - b) \Arrowvert_{2}^{2}
$$
  
\end{document}